\documentclass{article}
\usepackage[utf8x]{inputenc}
\usepackage[russian]{babel}
\usepackage{listings}
\usepackage{amsmath,amssymb}
%\usepackage{concmath}
%\usepackage{euler}

\numberwithin{equation}{section}

\begin{document}

\author{Дмитрий Джус}
\title{Курсовая работа по теме\\ «Обыкновенные дифференциальные уравнения»}
\maketitle

\newpage

\tableofcontents

\newpage
\part{Постановка задачи}

\section{Математический аспект}

Настоящая курсовая работа посвящена построению приближённого решения
краевой задачи для дифференциального уравнения вида

\begin{equation}\label{deq}
  \frac{d^2 u}{dx^2} + n(x) u = 0
\end{equation}

По условию, $n(x)$ представляет собой кусочно-непрерывную функцию,
равную константе $k^2$ на интервалах $(-\infty; 0)$ и $(a; +\infty)$ и
непрерывной известной функции на $[0; a]$. На границах $[0; a]$
решение и его первая производная должны удовлетворять условиям
сшивания:

\begin{subequations}\label{conds}
  \begin{align}
    u(0-0)& = u(0+0)&
    u(a-0)& = u(a+0)& \\
    u'(0-0)& = u'(0+0)&
    u'(a-0)& = u'(a+0)&
  \end{align}
\end{subequations}

\section{Физический аспект}

Задача соответствует возбуждения плоского слоя неоднородной
немагнитной среды электромагнитным полем, вектор электрического поля
распространяется вдоль оси $x$. $k$ имеет смысл волнового числа в
однородной части пространства, $n(x)$ — показатель преломления
неоднородной среды.

Из физических соображений, решение справа от слоя $[0; a]$ должно
иметь вид $u(x) = Be^{ikx}$, а слева — $u(x) = e^{ikx} +
Ae^{-ikx}$. Задача состоит в приближённом построении решения внутри
неоднородного слоя и отыскании комплексных констант $A$ и $B$
(называемых, соответственно, коэффициентами отражения и прохождения
электромагнитного поля).

Действительная область значений функции $n(x)$ соответствует
отсутствию поглощения энергии в среде, так что критерием правильности
полученных приближённых значений служит энергетическое тождество:

\begin{equation*}
  |A|^2 + |B|^2 = 1
\end{equation*}

\newpage
\part{Решение}

\section{Общие замечания}

Предлагаемые к реализации в курсовой работае приближённые методы были
описаны на алгоритмическом языке Scheme (простом и строго
функциональном диалекте Лиспа).

\section{Решение методом построения фундаментальной матрицы}

\subsection{Описание}

Дифференциальное уравнение второго порядка \eqref{deq} можно представить
в матричной форме, введя $v(x) = \frac{du}{dx}$:

\begin{equation}\label{mdeq}
  \frac{d}{dx}
  \begin{pmatrix}
    u(x) \\
    v(x)
  \end{pmatrix}
  = \begin{pmatrix}
    0 & 1 \\
    -n(x) & 0
  \end{pmatrix}
  \begin{pmatrix}
    u(x) \\
    v(x)
  \end{pmatrix}
\end{equation}

Решение системы \eqref{mdeq} с матрицей $A(x) = \bigl( \begin{smallmatrix} 0
  & 1 \\ -n(x) & 0 \end{smallmatrix} \bigr)$ можно найти в общем виде, если
построена фундаментальная матрица $\Omega(x)$:

\begin{equation}\label{fm}
\Omega(x) = \begin{pmatrix}
  \omega_{11}(x) & \omega_{12}(x) \\
  \omega_{21}(x) & \omega_{22}(x)
\end{pmatrix}
\end{equation}

$\Omega(x)$ является решением следующей матричной задачи Коши:

\begin{equation}\label{fmeq}
  \frac{d}{dx} \Omega(x) = A(x) \Omega(x)
\end{equation}

с условием $\Omega(0) = \bigl( \begin{smallmatrix} 1& 0\\ 0&
  1\end{smallmatrix} \bigr)$.

Решение \eqref{mdeq} для $\forall x \in [0; a]$ записывается в виде

\begin{equation}\label{fmsolution}
\begin{pmatrix}
  u(x) \\
  v(x)
\end{pmatrix} = 
\begin{pmatrix}
  \omega_{11}(x) & \omega_{12}(x) \\
  \omega_{21}(x) & \omega_{22}(x)
\end{pmatrix}
\begin{pmatrix}
  1 + A \\
  ik(1 - A)
\end{pmatrix}
\end{equation}

Использование краевых условий \eqref{conds} позволяет получить
следующую линейную алгебраическую систему относительно неизвестных $A$
и $B$:

\begin{equation}\label{abeq}
  \begin{split}
    (\omega_{11}(a) - \omega_{12}(a) ik) A - e^{ika} B &= -ik
    \omega_{12}(a) - \omega_{11}(a) \\
    (\omega_{21}(a) - \omega_{22}(a) ik) A - ike^{ika} B &= -ik
    \omega_{22}(a) - \omega_{21}(a)
  \end{split}
\end{equation}

Нахождение значений компонентов матрицы $\Omega$ на правом конце $[0;
a]$, таким образом, позволит получить искомые значения $A$ и $B$.

Для поиска фундаментальной матрицы в точках $[0; a]$ предлагается
использование следующего численного метода, основанного на приближении
функции $n(x)$ кусочно-постоянной, то есть замене неоднородной среды
на плоско-слоистую с большим числом однородных слоёв.

Рассматриваемый интервал разбивается на $N$ интервалов точками
$x_i = \frac{a}{N}i$, $i = 1 \ldots N$, в середине каждого из
которых берётся точка $\overline{x_i} = \frac{a}{N}(i+\frac{i}{2})$. В
данной точке функция $n(x)$ аппроксимируется ступенчатой функцией
$\overline{n}(x) = n(\overline{x_i})$, так что \eqref{mdeq} обращается
в систему с постоянной матрицей $A_i = \bigl(
\begin{smallmatrix}0& 1 \\ -\overline{n}(x)& 0 \end{smallmatrix}
\bigr)$. Тогда на каждом интервале $[x_i; x_{i+1}]$ решение
\eqref{fmeq} может быть найдено явно в виде матричной экпоненты:

\[
\Omega_i(x) = e^{A_i (x - x_i)}
\]

При этом, значения фундаментальной матрицы $\Omega_N(x_{i+1})$ на
правом конце $[x_i; x_{i+1}]$ служат начальным условием задачи Коши на
$[x_{i+1}; x_{i+2}]$, так что для каждой точки $\hat{x} \in [x_i;
x_{i+1}]$ фундаментальная матрица \eqref{fm} имеет вид:

\[
\Omega_N(\hat{x}) = e^{A_i(\hat{x}-x_i)} e^{A_{i-1}(x_i-x_{i-1})} \ldots e^{A_0(x_1-x_0)}
\]

На правом конце интервала $[0; a]$ приближённая фундаментальная
матрица записывается в виде:

\begin{equation}\label{fma}
  \Omega_N(a) = e^{A_{N-1}(x_N-x_{N-1})} e^{A_{N-2}(x_{N-1}-x_{N-2})} \ldots e^{A_0(x_1-x_0)}
\end{equation}

Для нахождения матричной экспоненты используется формула Тейлора:

\begin{equation}
\begin{split}
  e^{A_i(x-x_i)} \approx E +& \frac{1}{1!}{A_i(x-x_i)} +
  \frac{1}{2!}{A_i^2(x-x_i)^2}\\
  +& \frac{1}{3!}{A_i^3(x-x_i)^3} + \frac{1}{4!}{A_i^4(x-x_i)^4}
\end{split}
\end{equation}

После построения фундаментальной матрицы в $a$ коэффициенты $A$ и $B$
находятся из \eqref{abeq} с использованием полученных компонент
$w_{ij}^N(a)$.

Решение на каждом отрезке разбиения $[0; a]$ находится из
\eqref{fmsolution} с использование найденного $A$ и последовательности
приближений фундаментальной матрицы на $[x_i; x_{i+1}]$ для $\forall
i$.

\subsection{Реализация}

\subsection{Результаты}

\input{fm-solution}

\section{Решение методом последовательных приближений}

\subsection{Описание}

\subsection{Реализация}

\subsection{Результаты}

\newpage
\appendix
\part{Исходные тексты}

\input{listing}

\end{document}